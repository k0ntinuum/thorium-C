

\documentclass{article}
\usepackage[utf8]{inputenc}
\usepackage{setspace}
\usepackage{ mathrsfs }
\usepackage{graphicx}
\usepackage{amssymb} %maths
\usepackage{amsmath} %maths
\usepackage[margin=0.2in]{geometry}
\usepackage{graphicx}
\usepackage{ulem}
\setlength{\parindent}{0pt}
\setlength{\parskip}{10pt}
\usepackage{hyperref}
\usepackage[autostyle]{csquotes}

\usepackage{cancel}
\renewcommand{\i}{\textit}
\renewcommand{\b}{\textbf}
\newcommand{\q}{\enquote}
%\vskip1.0in





\begin{document}

{\setstretch{0.0}{


\b{Thorium} is a specialized binary version of Cesium.. It uses an $n \times n$  matrix of bits as a key. The trace of the matrix is added (mod 2) to the plaintext bit to get the ciphertext bit. The key matrix is adjusted as a function of this ciphertext bit. Rows and columns are circular-shifted, depending on the trace of the current key and the latest ciphertext symbol.

}}
\end{document}
